%%%%%%%%%%%%%%%%%%%%%%%%%%%%%%%%%%%%%%%%%%%%%%%%%%%%%%%%%%%%%%%%%%%%%%%%%%%%%%%%
%2345678901234567890123456789012345678901234567890123456789012345678901234567890
%        1         2         3         4         5         6         7         8

\documentclass[letterpaper, 10 pt, conference]{ieeeconf}  % Comment this line out if you need a4paper

%\documentclass[a4paper, 10pt, conference]{ieeeconf}      % Use this line for a4 paper

\IEEEoverridecommandlockouts                              % This command is only needed if 
                                                          % you want to use the \thanks command

\overrideIEEEmargins                                      % Needed to meet printer requirements.

% See the \addtolength command later in the file to balance the column lengths
% on the last page of the document

% The following packages can be found on http:\\www.ctan.org
%\usepackage{graphics} % for pdf, bitmapped graphics files
%\usepackage{epsfig} % for postscript graphics files
%\usepackage{mathptmx} % assumes new font selection scheme installed
%\usepackage{times} % assumes new font selection scheme installed
%\usepackage{amsmath} % assumes amsmath package installed
%\usepackage{amssymb}  % assumes amsmath package installed

\title{\LARGE \bf
Optimization-based Motion Retargeting Integrating Spatial and Dynamic Constraints
}


\author{Thomas Moulard$^{1}$, Eiichi Yoshida$^{1}$ and Shin'ichiro
  Nakaoka$^{2}$% <-this % stops a space
\thanks{*This work was not supported by any organization}% <-this % stops a space
\thanks{$^{1}$T.~Moulard and E.~Yoshida are with CNRS-AIST JRL (Joint
  Robotics Laboratory), UMI3218/CRT, and $^{1}$S.~Nakaoka are with
  Humanoid Research Group, both National Institute of Advanced Industrial Science and Technology (AIST)
Tsukuba Central 2, 1-1-1 Umezono, Ibaraki 305-8568 Japan
        {\tt\footnotesize \{thomas.moulard, e.yoshida, s.nakaoka\}@aist.go.jp}}%
}


\begin{document}



\maketitle
\thispagestyle{empty}
\pagestyle{empty}


%%%%%%%%%%%%%%%%%%%%%%%%%%%%%%%%%%%%%%%%%%%%%%%%%%%%%%%%%%%%%%%%%%%%%%%%%%%%%%%%
\begin{abstract}
In this paper we present an optimization-based retargeting method for
precise reproduction of captured human motions by a humanoid robot.
We take into account two important aspects of retargeting
simultaneously: spatial relationship and robot dynamics model.
The former takes care of the spatial relationship between the body
parts based on ``interaction mesh'' to follow the human motion in a
natural manner, whereas the latter adapts the resulting motion in such
a way that the dynamic constraints such as torque limit or dynamic
balance are satisfied. 
We have integrated the interaction mesh and the dynamic constraints in an
optimization framework, which is advantageous for generation of
natural motions by a humanoid robot compared to 
previous work that performs those processes separately. The retargeted
motions are validated through dynamic simulations of a humanoid robot.
\end{abstract}


%%%%%%%%%%%%%%%%%%%%%%%%%%%%%%%%%%%%%%%%%%%%%%%%%%%%%%%%%%%%%%%%%%%%%%%%%%%%%%%%
\section{Introduction}
\label{sec:intro}

One of the advantages of human-size humanoid robots is
its ability to generate whole-body motions maintaining similar
dynamics to humans. 
This ability allows a humanoid robot to serve as an entertainer like
a dancer of an actor \cite{nakaoka_iros2010}, or also to use various
machines and devices designed for humans \cite{Yokoi03iros}.
As an extension of the latter use, a new application
has recently been studied: a humanoid robot as an evaluator of human
assistive devices \cite{Takanishi06ICRA,Miura13ICRA}.
If a humanoid reproduces human motions faithfully, it can be used to
test the devices instead of human subjects. 
This brings several benefits such as no need for ethical 
process, repeated test with exactly the same motions under the same
conditions, and qualitative evaluation through sensory measurement
like torque and force.
It has been demonstrated that the human-size humanoid HRP-4C
\cite{Kaneko09Humanoids}  can evaluate the effect of load reduction
quantitatively by estimating motor torque \cite{Miura13ICRA}, taking
an example of a supportive wear called ``Smart Suit Lite''
\cite{Tanaka11JRM} designed to reduce load 
at lower back with embedded elastic bands. 

The important issue in those applications is how to generate natural
motions of a humanoid robot. 
There have been a number of studies on ``motion retargeting''
techniques in order to
generate humanoid motions based on those of a human measured by a
motion capture system.
Regargeting captured motion to humanoids has been
actively studied during last decade, thanks to the progress of their
dynamic capability. The work of Pollard \cite{Pollard02ICRA} is one of
the pioneering studies that enable reproduction of human motions
by a humanoid, in this case the upper body of Sarcos humanoid robot,
by taking into account various constraints. 
Nakaoka et al. developed a technique to  transfer
Japanese traditional dancing motions  to a humanoid by introducing a
notion of leg task model \cite{nakaoka_icra2004,nakaoka_2007}.
Miura et al. \cite{Miura11IROS} devised a walking pattern generator
that allows the humanoid robot HRP-4C to walk in a manner very close
to humans, including stretched knees, swing-leg trajectory and single
support phase on the toe.
Other imitation methods have been proposed based on a dynamic
controller  \cite{Yamane11humanoids,Ramos11humanoids}, 
motion recognition and primitives \cite{Ott08humanoids} and
extraction of upper-body motion from markerless motion input
\cite{Dariush08IROS,Do08humanoids}. 



% Human motion based optimization
% Categoraized related work
% Takano
% \begin{itemize}
% \item Pollard (upper body), Dariush, Lee, Kulic, Yamane, Miura, Mansard
%   (walking) Martin Do-KIT
% 
% \item Manocha, Kallman: support for humanoid animation -- graphic only
% \item Nakaoka humanoids 2012 -- graphic + robot
% \item Motion optimization Miossec, Suleiman, Lengagne
% \end{itemize}


On the other hand, motion retargeting has been investigated
intensively in computer graphics domain, typically to generate motions
for new characters based on motion capture data using space-time
constraints solver \cite{Gleicher98}.  
Recently, Ho et al. proposed a new retargeting method called
``interaction mesh'' 
that preserves the spatial relationship between closely interacting body
parts and the objects in the environments \cite{Komura10}.
Nakaoka and Kokura extended this method for retargeting to a humanoid
robot by taking advantage of its capacity to adapt motions
to a character with very different physical properties
\cite{Nakaoka12Humanoids}. Usage of interaction mesh brings natural
following of original human motion and self-collision avoidance.
Although this method includes quasi-static
balance consideration by shifting the waist, it does not deal with 
dynamic constraints such as torque limits or dynamic ZMP-based
stability. In addition, those constraints are treated separately after
generating retargeted motion to adapt to the humanoid.


% Optim
Another related domain is the optimization technique that is more and
more employed to generate robot trajectories minimizing certain cost
function under mechanical or dynamic constraints.
Miossec et al. applied nonlinear optimization dynamic whole-body motion
like a kicking motion of a human-size humanoid \cite{Miossec06ROBIO}.
Recently the optimization is utilized for generation of multi-contact
dynamic motion through modeling of dynamic constraints using Taylor
expansion \cite{Lengagne13IJRR}.
Suleiman et al. proposed another trajectory optimization technique
based on Lie 
algebra that allows efficient computation through analytic integration
of dynamics \cite{Suleiman07Humanoids} and applied it to human motion
imitation \cite{Suleiman08ICRA}. 
The latter research aims at optimizing the humanoid trajectory to be as
close as human motion, but self-collision avoidance is incorporated as
a post-processing to the optimized motion like previous work
\cite{Nakaoka12Humanoids}. This is disadvantageous 
because separate application of collision avoidance may lead to
unnatural motions. 


% Motivation and contributions
% Problem: dynamic constraints not well considered
% To our knowledge, there has been no motion optimization integrating
% motion retargeting and self-collision or interaction with
% environment at the same time.
Methods in previous research are therefore still lacking 
capability to optimize humanoid motions by taking into account 
retargeting, dynamics constraints and
self-collision at the same time, in
order to create the humanoid motion as close to human motions as possible.
In this paper, we address this retargeting by formulating it
as a nonlinear optimization problem upder spatial and
dynamic constraints.
Our contributions are...
The paper is organized as follows...

\section{Method Overview}
\label{sec:overview}


\section{Conclusions}

This is conclusion.

% \addtolength{\textheight}{-12cm}   % This command serves to balance the column lengths
                                  % on the last page of the document manually. It shortens
                                  % the textheight of the last page by a suitable amount.
                                  % This command does not take effect until the next page
                                  % so it should come on the page before the last. Make
                                  % sure that you do not shorten the textheight too much.

%%%%%%%%%%%%%%%%%%%%%%%%%%%%%%%%%%%%%%%%%%%%%%%%%%%%%%%%%%%%%%%%%%%%%%%%%%%%%%%%



%%%%%%%%%%%%%%%%%%%%%%%%%%%%%%%%%%%%%%%%%%%%%%%%%%%%%%%%%%%%%%%%%%%%%%%%%%%%%%%%



%%%%%%%%%%%%%%%%%%%%%%%%%%%%%%%%%%%%%%%%%%%%%%%%%%%%%%%%%%%%%%%%%%%%%%%%%%%%%%%%
\section*{Appendix}

Appendixes should appear before the acknowledgment.

\section*{Acknowledgment}

JSPS Fellowship

%%%%%%%%%%%%%%%%%%%%%%%%%%%%%%%%%%%%%%%%%%%%%%%%%%%%%%%%%%%%%%%%%%%%%%%%%%%%%%%%


\bibliographystyle{IEEEtran}
\bibliography{yoshida-main,yoshida-ca,humanoid,assist,graphics,jrl}


\end{document}
